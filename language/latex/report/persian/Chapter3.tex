\chapter{
جداول در
\lr{LaTeX}
}
\label{chap:chap3}

شاید طراحی جدول در
\lr{LaTeX}
جزو کارهای سخت به نظر برسد. البته در حقیقت اینطور نیست. اگر شما روی این نظر مصر هستید و در این زمینه
\lr{LaTeX}
را با نرم افزارهای ویرایش متن
\lr{microsoft}
مقایسه می‌کنید، باید به شما بادآوری کنیم که هر چه یک رابط گرافیکی کارتان را راحت کند، به همان میزان برای شما تصمیم می‌گیرد و شما را از یک «توسعه‌دهنده» بودن، به سمت یک یک «کاربر عادی» بودن سوق می‌دهد.

در این متن، من کارهایی که باید بکنید تا در
\lr{LaTeX}
جدول درست شود را توضیح نمی‌دهم. خودتان اسکریپت‌ها را ببینید تا بفهمید. اما اولین توصیه من برای طراحی جدول، همان چیزی است که در قسمت
\ref{sec:text} 
گفتم. سعی کنید در
\lr{LaTeX}
، چه فارسی چه انگلیسی، هر دستور را در یک سطر مجزا بنویسید. آن موقع خواهد فهمید که سخت‌ترین کارها چقدر آسان می‌شود. باید توجه کنید که وقتی با
\lr{LaTeX}
متن می‌نویسید، شما در واقع به همان میزان که دارید تولید محتوی می‌کنید، دارید کد هم می‌زنید! و خوب اگر یک برنامه‌نویس باشید، می‌دانید که جدای از الگوریتمی که اجرا می‌شود، نجوه نگارش متن برنامه چقدر مهم است؛ و چقدر در فهم و راحت‌شدن کار کمک می‌کند. 

جدول
\ref{tab:inventory}
یک نمونه برای شما است. نگاه کنید که آن را چطور نوشته‌ام. سعی کنید خوب فرق عملگرهای مربوط به سطر جدید یا ستون جدید را بفهمید.


\begin{table}[H] 
\caption{کپشن برای جدول}
\label{tab:inventory}
\begin{tabularx}{\textwidth}{| X | X |}
    \hline
     مورد
      & 
      توضیح
        \\ \hline
      
     مورد ۱
     & 
     توضیح ۱ 
     \\ \hline
     
     مورد ۲ 
     & 
     توضیح ۲ 
     \\ \hline
     
     مورد ۳ 
     & 
     توضیح ۳
     \\ \hline
     
\end{tabularx}
\end{table}


یک حقه ساده دیگر که البته این یکی را اگر تا کنون با
\lr{LaTeX}
کار کرده باشید به احتمال زیاد می‌دانید. می‌توانید برای راحتی در طراحی جدول، از وبگاه‌هایی که به صورت رایگان و با رابط گرافیکی اسکریپت
\lr{LaTeX}
مربوط به جدول دلخواه شما را تولید می‌کنند استفاده کنید. خیلی ساده است. کافی است عبارت زیر را در گوگل جستجو کنید:

\begin{flushleft}
\lr{latex table online}
\end{flushleft}

یک جدول نمونه دیگر:

\begin{table}[!ht]
\caption{
اطلاعات مربوط به قفسه‌ها و انبار شبیه‌سازی‌شده
}
\label{tab:sim_shelf_info}
\centering
\begin{tabular}{|c|c|}
\hline
مولفه                     & اندازه \\ \hline
ابعاد آرایه سلول‌های قفسه &    6 × 10    \\ \hline
ابعاد کل قفسه             &  \lr{m} 0.4 × \lr{m} 3.77 × \lr{m} 3.5 \\ \hline
ابعاد هر سلول             &  \lr{cm} 40 × \lr{cm} 57 × \lr{cm} 30 \\ \hline
ابعاد برچسب‌های رنگی      &  \lr{cm} 5 × \lr{cm} 5 \\ \hline
ابعاد نشانگرها            & \lr{cm} 10 × \lr{cm} 10 \\ \hline
فاصله میان دو قفسه متوالی &  \lr{m} 3 \\ \hline
تعداد کل قفسه‌ها          &   5   \\ \hline
\end{tabular}
\end{table}
