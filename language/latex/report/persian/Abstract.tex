
\section*{چکیده}
% سطر بعدی برای آمدن چکیده در فهرست است. به نظرم کار مناسبی نیست.
%\addcontentsline{toc}{section}{چکیده}
\vspace{2cm}

\large
قالب حاضر، برای تسهیل در نگارش متن در بستر
\lr{LaTeX}
تهیه شده است. 
در هر فصل، آموزش فارسی یک سری از کارهای
\lr{LaTeX}
ارائه شده است. در واقع متن این سند شامل مجموعه‌ای از توصیه‌ها است. اما آموزش اصلی با رجوع شما به اسکریپت تولیدکننده هر بخش از متن انجام می‌گیرد. به این ترتیب، با رفتن به سراغ هر یک از فصل‌های مربوطه، در متن فارسی توصیه‌هایی کارآ را می‌خوانید و در کد
\lr{compile}
شده به فرمت «
\lr{.tex}
»، مواد لازم برای ایجاد فصل‌بندی، فهرست‌های مختلف، قراردادن اشکال و جداول و ... را می‌یابید. به قسمت‌های مختلف این نوشتار رجوع کنید و اسکریپت هر قسمت را ببینید تا نحوه انجام کار مربوطه دستتان بیایبد. به عنوان مثال، این قسمت مربوط به چکیده است.
در اینجا چکیده متن در حدود ۳۰۰ کلمه نوشته می شود. 
\\

\vspace{2cm}

\noindent \textit{
 کلمات کلیدی:
 کلمه ۱,
  کلمه ۲,
  کلمه ۳}