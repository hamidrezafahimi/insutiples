\chapter{
قراردادن روابط ریاضی
}

در ادامه، روابط ریاصی در این فصل قرار می‌گیرد. قراردادن روابط ریاضی نیز بسیار ساده است. در اینجا دو نوع رابطه ریاضی، روابط تنها و دسته روابط بررسی می‌شوند.


\section{ 
قراردادن یک رابطه ریاضی تنها
}
\label{sec:singleEq} 

یک نمونه از معادله تنها به صورت معادله
\ref{eq:rotMat}
است.

\begin{equation}
\label{eq:rotMat}
R = \begin{bmatrix} r_{11} & r_{12} & r_{13} \\ r_{21} & r_{22} & r_{23} \\ r_{31} & r_{32} & r_{33} \end{bmatrix}
\end{equation}

این هم چند نمونه معادله دیگر، برای آنها که وقت ندارند و کپی‌کردن را به مسلط شدن ترجیچ می‌دهند.

\begin{equation}
\label{eq:pointTransform}
\begin{bmatrix} X \\ Y \\ Z \end{bmatrix}^o = [\overrightarrow{T}]_c^o + [R]_c^o \begin{bmatrix} x \\ y \\ z \end{bmatrix}^c
\end{equation} 

\begin{equation}
\label{eq:rotationMatrix}
R = \cos(\alpha)I + (1-\cos{\alpha})rr^T + \sin(\theta)\begin{bmatrix} 0 & -r_z & r_y \\ r_z & 0 & -r_x \\ -r_y & r_x & 0 \end{bmatrix} 
\end{equation}

\begin{equation}
\label{eq:AHs}
A_{RH} = A_{LH} = A_{UH} = A_{DH} = \frac{A_{total}}{2}
\end{equation}

\begin{equation}
%\begin{aligned}
areaRatio = \frac{smallestTag}{largestTag}
\label{eqn:areaRatio}
%\end{aligned}
\end{equation}

\begin{equation}
\label{eq:co-row standard}
\begin{aligned}
if:  \;\;\; &|y_P-y_Q|<\Delta y_{lim} \;\;\; and  \;\;\; \Delta x_{lim}<|x_P-x_Q|  \;\;\; and  \;\;\; |x_P-x_Q|=\Delta x_{min} \;\;\; \\ &\Rightarrow   \;\;\; P,Q\in R_k 
\end{aligned}
\end{equation}

\begin{equation}
\label{eq:co-column standard}
\begin{aligned}
if:  \;\;\; &|x_P-x_Q|<\Delta x_{lim} \;\;\; and  \;\;\; \Delta y_{lim}<|y_P-y_Q|  \;\;\; and  \;\;\; |y_P-y_Q|=\Delta y_{min} \;\;\; \\ &\Rightarrow   \;\;\; P,Q\in C_k 
\end{aligned}
\end{equation}

\section{ 
دسته روابط
}
\label{sec:Introduction} 


یک نمونه از معادله با دو زیرمعادله به صورت معادلات
\ref{eq:euler_from_rotmat}
می‌باشد. این هم یک نمونه توضیح مولفه‌های موجود در رابطه:

در برابری
\ref{eq:euler_from_rotmat}
،
\lr{$\phi_{v}$}
،
\lr{$\theta_{v}$}
و
\lr{$\psi_{v}$}
به ترتیب به زاویه چرخش حول محور
\lr{x}
، پیچش حول محور
\lr{y}
و گردش حول محور
\lr{z}
اشاره دارند و زیروند
\lr{v}
حکایت از تحصیل این سه مولفه از طریق ناوبری تصویری دارد.

\begin{subequations}
\label{eq:euler_from_rotmat}
\begin{align}
\phi_{v} &= atan2(-r_{32}, r_{33}) \\
\theta_{v} &= asin(r_{31}) \\
\psi_{v} &= atan2(-r_{21}, r_{11})
\end{align}
\end{subequations}
 
 و این هم چندتای دیگر
 
 \begin{subequations}
\label{eq:rodriguesVec}
\begin{align}
\alpha &= norm(\overrightarrow{V}) = \sqrt{v_1^2 + v_2^2 + v_3^2} \\
\overrightarrow{r} &= \frac{1}{\alpha} \begin{bmatrix} v_1 \\ v_2 \\ v_3 \end{bmatrix}
\end{align}
\end{subequations}


\begin{subequations}
\label{eq:avoid_guidance}
\begin{align}
\dot y &= ((\frac{A_{obs,R}}{A_{RH}})-(\frac{A_{obs,L}}{A_{LH}}))\dot y_{max} \\
\dot z &= ((\frac{A_{obs,U}}{A_{UH}})-(\frac{A_{obs,D}}{A_{DH}}))\dot z_{max}
\end{align}
\end{subequations}



\begin{subequations}
\label{eq:dynamic_model}
\begin{align}
\ddot{x} &= (\cos{\phi}\sin{\theta}\cos{\psi}+\sin{\phi}\sin{\psi})\frac{1}{m}U \\
\ddot{y} &= (\cos{\phi}\sin{\theta}\sin{\psi}+\sin{\phi}\cos{\psi})\frac{1}{m}U \\
\ddot{z} &= -g + (\cos{\phi}\cos{\theta})\frac{1}{m}U \\
\ddot\phi &= \dot\theta\dot\psi\frac{I_y-I_z}{I_x} + \frac{J_r}{I_x}\dot\theta\Omega + \frac{l}{I_x}M_{\phi} \\
\ddot\theta &= \dot\phi\dot\psi\frac{I_z-I_x}{I_y} + \frac{J_r}{I_y}\dot\phi\Omega + \frac{l}{I_x}M_{\theta} \\
\ddot\psi &= \dot\phi\dot\theta\frac{I_x-I_y}{I_z} \frac{l}{I_z}M_{\psi}
\end{align}
\end{subequations}