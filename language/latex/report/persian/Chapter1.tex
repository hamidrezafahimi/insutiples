\chapter{
متن و ارجاعات
}
\label{chap1}

انگیزه اصلی من برای تهیه این قالب زمانی شکل گرفت که از ویندوز به لینوکس مهاجرت کردم. خوب، برای من تقریبا هیچ کدام از امکانات ویندوز یک ضرورت اساسی به شمار نمی‌آمد. تنها چیزی که باقی مانده بود، نرم‌افزارهای ویرایش متن ویندوز بودند. قبلا با
\lr{LaTeX}
کار کرده بودم. اما حالا نیاز داشتم که قالب‌هایی را برای کارهای مختلف خود یا همکارانم آماده کنم. می‌دانستم که قسمت سخت بستر
\lr{LaTeX}
، همین قالب نویسی آن است. بعد از آن، کارها به مراتب از ویندوز راحت‌تر خواهد بود. بنابراین شروع کردم. چند قالب آماده از وبگاه
\lr{Overleaf}
دانلود کردم و آنها را برای فارسی‌نویسی
\lr{customize}
کردم. نوشتار حاضر، قالب دلخواه نوشتن گزارش در بستر
\lr{LaTeX}
به زبان فارسی است. می‌توانید به راحتی آن را برای گزارش‌های مختلف تغییر دهید. از لوگوی دانشگاه خودمان در سربرگ اصلی استفاده کردم تا امکان قراردادن لوگو هم در قالب قراهم باشد. وگرنه این کاملا متن باز است. این شما و این
\lr{LaTeX}
.


در این قالب، برای راحتی، نمونه‌هایی از شکل، جدول و معادله آورده شده است. برای جزئیات بیشتر در مورد هر کدام، به راحتی می‌توانید با جستجو در اینترنت به اطلاعات مفیدی دست یابید. 

\section{ 
چند توصیه در مورد متن
}
\label{sec:text} 

با وجود آن که در اینجا فارسی می‌نویسیم، در اسکریپت‌نویسی
\lr{LaTeX}
، بسیار لازم می‌شود که در خلال متن فارسی، از متن انگلیسی استفاده کنیم، یا مطابق معمول در
\lr{LaTeX}
، از دستوراتی در خلال متن اصلی استفاده کنیم. در چنین شرایطی، توصیه می‌کنم که حتما متن فارسی و انگلیسی را در سطرهای مجزا قرار دهید. زیرا راست‌چین بودن فارسی و چپ‌چین بودن انگلیسی باعث می‌شود که متن اسکریپت به هم ریخته شود و خودتان یا کس دیگر نتوانید آن دوباره به راحتی بخوانید یا ویرایش کنید. به نحوه استفاده من از کلمات انگلیسی یا دستورات در خلال نوشتن به زبان فارسی توجه کنید.

برای این که در نوشتن راحت باشید، یک جقه ساده به شما پیشنهاد می‌دهم. متن فارسی خود را بنویسید و کامل کنید و هرجا که نیاز به متن انگلیسی یا دستوری در وسط متن بود یک خط خالی در وسط متن جا بگذارید. سپس در انتها متن‌ها یا دستورات مناسب را در جای خود قرار دهید.

\noindent
به صورت پیش‌فرض، در بندهای جدید، متن به اصطلاح
‌\lr{indent}
می‌شود. اما در این متن به صورت دستی
‌\lr{indent}
لغو شده است. به اسکریپت نگاه کنید و یاد بگیرید چطور انجام می‌شود.


\section{ 
ارجاعات
}
\label{sec:cross_refs}

چند جور ارجاع در
\lr{LaTeX}
داریم. پس به صورت زیر تقسیم می‌کنم که هم مرتب باشد و هم بدانید که می‌توانید برای بخش‌های فهرست متن خود «زیربخش» هم تعریق کنید.

\subsection{
ارجاعات درون متن
}
در این بحش نمونه ارجاع به قسمت‌های دیگر را می‌بینید. مثلا در قسمت
\ref{sec:text} 
در مورد متن‌نویسی توصیه‌های کوچکی مطرح کردم. در ادامه، در فصل
\ref{chap:chap2}
در مورد عکس‌ها نمونه آورده‌ام. شکل
\ref{fig:clock_tower_photo}
یک نمونه از آن است. جدول
\ref{tab:inventory}
در فصل
\ref{chap:chap3}
نمونه‌ای از جدول است. به اسکریپت این قسمت نگاه کنید تا ببینید چقدر ساده می‌توانید به هرجا که دوست داشتید ارجاع درون متن بدهید. تازه، می‌توانید روی عدد ارجاع‌ها بزنید تا به مقصد آن ارجاع هدایت شوید.

\subsection{ارجاع به منابع و آثار دیگر}
کلا اینجور کارها در
\lr{LaTeX}
ساده‌تر است. مثلا من الان به
\cite{romero2018speeded}
ارجاع می‌دهم. بعد هم اگر دلم بخواهد، به
\cite{ababsa2004robust, sahawarehouse, chapman2016computer}
همزمان ارجاع می‌دهم. به همین راحتی! قبل از این که بروید این مقاله‌ها را ببینید، اسکریپت من را نگاه کنید تا دستتان بیاید که ارجاع دادن چگونه است. بادتان باشد که اگر در متنتان ارجاعاتی به منابع وجود دارد، لازم است  فایل مربوطه‌ای با پسوند
\lr{'.bib'}
داشته باشید که با استفاده از آن بخش فهرست مراجع فعال شود. به اسکریپت اصلی این گزارش (
\lr{'main.tex'}
) نگاه کنید و
\lr{'references.bib'}
را در آن پیدا کنید تا کاملا روشن شود.

برای آن که متن مربوط به هر مرجع را در فایل
\lr{'references.bib'}
بنویسید هم راه‌های زیادی هست. مثلا وقتی در
\lr{Google Scholar}
جستجو می‌کنید، می‌توانید متن لازم برای فرمت
\lr{bibtex}
را در کنار فرمت‌های شناخته‌شده
\lr{APA}
و
\lr{MLA}
بیابید.

جدای از این حرف‌ها، به متن فایل
\lr{'references.bib'}
نگاهی بیاندازید. به راحتی خواهید توانست برای هر مرجع دلخواه داده‌های لازم را وارد کنید.
\lr{LaTeX}
بقیه کارها را برای شما انجام می‌دهد.

\subsection{پاورقی‌ها}
مثلا کلمه
، فاصله‌یاب های لیزری
\LTRfootnote{\lr{LIDAR}}
نمونه‌ای از یک پاورقی در یک متن فارسی است که معادل انگلیسی یک کلمه در آن آورده شده است.
همچنین،
\lr{ROS}
\footnote{
\lr{Robot Operating System}
یا اصطلاحا «سامانه عامل ربات»، یک مجموعه میان‌افزار رباتیک منبع باز است که مجموعه‌ای از چارچوب‌ها برای توسعه نرم افزارهای به کار رفته در ربات‌ها را شامل شده و خدماتی از قبیل ارتباط با سخت‌افزار، کنترل سطح پایین دستگاه، انتقال پیام بین فرایندها و ...را ارائه می‌نماید. فرایندهای جاری در 
\lr{ROS}،
در یک معماری نموداری نشان داده می شوند که در آن پردازش در گره(\lr{node})هایی انجام می شود و داده‌ها در قالب پیام‌هایی میان گره‌ها رد و بدل می‌شوند.
}
می‌تواند بهانه‌ای برای آوردن یک پاورقی توضیحی باشد.

\subsection{ارجاع به نشانی‌های اینترنتی}

به راحتی می‌توانید در
\lr{LaTeX}
لینک‌های اینترنتی را قرار دهید. مثلا این نشانی
\lr{github}
من است:

\begin{flushleft}
\url{https://github.com/hamidrezafahimi}
\end{flushleft}

